\documentclass[11pt,]{article}
\usepackage[margin=1in]{geometry}
\newcommand*{\authorfont}{\fontfamily{phv}\selectfont}
\usepackage[]{mathpazo}
\usepackage{abstract}
\renewcommand{\abstractname}{}    % clear the title
\renewcommand{\absnamepos}{empty} % originally center
\newcommand{\blankline}{\quad\pagebreak[2]}

\providecommand{\tightlist}{%
  \setlength{\itemsep}{0pt}\setlength{\parskip}{0pt}} 
\usepackage{longtable,booktabs}

\usepackage{parskip}
\usepackage{titlesec}
\titlespacing\section{0pt}{12pt plus 4pt minus 2pt}{6pt plus 2pt minus 2pt}
\titlespacing\subsection{0pt}{12pt plus 4pt minus 2pt}{6pt plus 2pt minus 2pt}

\titleformat*{\subsubsection}{\normalsize\itshape}

\usepackage{titling}
\setlength{\droptitle}{-.25cm}

%\setlength{\parindent}{0pt}
%\setlength{\parskip}{6pt plus 2pt minus 1pt}
%\setlength{\emergencystretch}{3em}  % prevent overfull lines 

\usepackage[T1]{fontenc}
\usepackage[utf8]{inputenc}

\usepackage{fancyhdr}
\pagestyle{fancy}
\usepackage{lastpage}
\renewcommand{\headrulewidth}{0.3pt}
\renewcommand{\footrulewidth}{0.0pt} 
\lhead{}
\chead{}
\rhead{\footnotesize USP 597: Regional Economic Analysis -- Spring 2019}
\lfoot{}
\cfoot{\small \thepage/\pageref*{LastPage}}
\rfoot{}

\fancypagestyle{firststyle}
{
\renewcommand{\headrulewidth}{0pt}%
   \fancyhf{}
   \fancyfoot[C]{\small \thepage/\pageref*{LastPage}}
}

%\def\labelitemi{--}
%\usepackage{enumitem}
%\setitemize[0]{leftmargin=25pt}
%\setenumerate[0]{leftmargin=25pt}




\makeatletter
\@ifpackageloaded{hyperref}{}{%
\ifxetex
  \usepackage[setpagesize=false, % page size defined by xetex
              unicode=false, % unicode breaks when used with xetex
              xetex]{hyperref}
\else
  \usepackage[unicode=true]{hyperref}
\fi
}
\@ifpackageloaded{color}{
    \PassOptionsToPackage{usenames,dvipsnames}{color}
}{%
    \usepackage[usenames,dvipsnames]{color}
}
\makeatother
\hypersetup{breaklinks=true,
            bookmarks=true,
            pdfauthor={ ()},
             pdfkeywords = {},  
            pdftitle={USP 597: Regional Economic Analysis},
            colorlinks=true,
            citecolor=blue,
            urlcolor=blue,
            linkcolor=magenta,
            pdfborder={0 0 0}}
\urlstyle{same}  % don't use monospace font for urls


\setcounter{secnumdepth}{0}





\usepackage{setspace}

\title{USP 597: Regional Economic Analysis}
\author{Jamaal Green}
\date{Spring 2019}


\begin{document}  

		\maketitle
		
	
		\thispagestyle{firststyle}

%	\thispagestyle{empty}


	\noindent \begin{tabular*}{\textwidth}{ @{\extracolsep{\fill}} lr @{\extracolsep{\fill}}}


E-mail: \texttt{\href{mailto:jamgreen@pdx.edu}{\nolinkurl{jamgreen@pdx.edu}}} & Web: TBD\\
Office Hours: W 09:00-11:30 a.m.  &  Class Hours: M 11:00-01:15 p.m.\\
Office: 230A Brackett Hall  & Class Room: CUPA 225\\
	&  \\
	\hline
	\end{tabular*}
	
\vspace{2mm}
	


\section{Course Description}\label{course-description}

Thiscourse offers students an opportunity to learn about and apply tools
for analyzing regional economies, as well as, becoming better analysts
by applying the skills of data science workflows. The course will
combine video lecture, demonstration, on-line lab, and targeted
assignments to assist students learning about data concepts, applying
available regional data tools, developing regional economic data
products, and organizing data projects to maximize their portability and
reproducability.

\section{Objectives}\label{objectives}

This two-credit course builds on MURP core courses, especially Planning
Methods II (USP 535), and will complement other courses in the MURP
economic development pathway (USP 572, Regional economic development,
USP 517, Economic development policy, and USP 578, Impact Assessment),
by providing students skills important to economic development
practitioners and researchers. It is also appropriate for students in
other degree programs (eg, MUS, MRED, MPP) or graduate certificate
programs (eg, Applied Demography) looking for a focused discussion of
regional analysis methods and tools.

There are no course prerequisites, although students are strongly
encouraged to have completed a basic or applied statistics course, such
as USP 535 (Planning Methods I). It is also quite helpful to have some
basic excel skills. In addition to excel, students should have some
experience with the statistical programming language, R, or some other
scripting language (eg Python). While it is possible to complete this
course using only excel, being able to take advantage of scripting will
make some projects easier and help with overall project management. For
students who do not have experience with R, a subscription to
\href{https://www.datacamp.com/}{DataCamp} will be provided through the
course if enrollment exceeds 10 students. Again, experience with R
\emph{is not} required for the successful completion of this course.

Students will choose a regional economy that they will analyze over the
course of the quarter. Each week will build on the previous week by
providing students additional data sources and tools for adding an
additional layer of understanding of the region's economy. They will
also build data manipulation, management and visualization skills using
excel or R.

\section{Learning Outcomes}\label{learning-outcomes}

At the completion of the term, students should be able to:

\begin{itemize}
\item
  Locate, analyze, and interpret important sources and concepts of
  demographic and economic data;
\item
  Apply tools of demographic and economic analysis, including population
  pyramids, location quotients, shift share analysis, and cluster
  analysis;
\item
  Understand and apply methodologies for conducting Economic Opportunity
  Analysis as prescribed by the Oregon Statewide Land Use Program.
\item
  Apply good data project management and visualization practices.
\end{itemize}

\section{Requirements}\label{requirements}

\subsection{Required Texts and
Readings:}\label{required-texts-and-readings}

There is no required text. I will provide worksheets and information via
the course website. However, there are several good sources of basic
information regarding data sources, methods and programming guidance. I
recommend the following:

\begin{quote}
Quinterno, John. 2014. \emph{Running the Numbers: A Practical Guide to
Regional Economic and Social Analysis}. Armonk, NY: M.E. Sharpe.
\end{quote}

\begin{quote}
Wickham, Hadley and Grolemund, Garret. 2017. \emph{R For Data Science}.
O'Reilly Media. book website: \href{https://r4ds.had.co.nz/}{R4DS}
\end{quote}

\section{Method of Evaluation}\label{method-of-evaluation}

\begin{itemize}
\item
  Each week, students will complete an assignment that applies the data
  tools and concepts to their selected region. Students will be graded
  on each assignment, worth 9 points each.
\item
  The final ten percent will be based on participation in online
  conversation forums regarding the material and the assignments, for a
  total of 100 points as a course grade (see below).
\item
  \underline{Late Submission Policy}: Each weekly assignment will be due
  via email at 11:59 pm on Sunday of the relevant week. For each hour
  after the deadline, a 0.1 point deduction (from the 10 point total)
  will be applied, up to a maximum deduction of 5 points. • There is no
  final exam.
\end{itemize}

\section{Office Hours}\label{office-hours}

\begin{itemize}
\tightlist
\item
  \textbf{TBD}
\end{itemize}

\section{Course Overview}\label{course-overview}

\subsection{Week 1, 04/01-04/05}\label{week-1-0401-0405}

\begin{itemize}
\tightlist
\item
  Broad overview of data sources (i.e.~BLS, Census, BEA, state LMI
  agencies) and data concpets that are important to doing regional
  economic analysis
\end{itemize}

\textbf{FIRST ASSIGNMENT- DUE APRIL 8}- Choose Metro Area for Analysis

\subsection{Week 2, 04/08-04/12}\label{week-2-0408-0412}

\begin{itemize}
\item
  Review of basic demographic concepts (introduced in USP 535): natural
  increase, net migration, race and ethnicity, fertility and mortality,
  families and households etc
\item
  Apply demographic analysis tools in relation to economic development,
  such as population pyramids and projects, and migration analyis.
\end{itemize}

\textbf{SECOND ASSIGNMENT- DUE APRIL 15}- Regional Demographic Analysis

\section{Incomplete Grades}\label{incomplete-grades}

I am very reluctant to assign incomplete grades, and will only do so
when circumstances are consistent with PSU's policy. Poor planning is
not a valid excuse. If you encounter unforeseen circumstances that meet
the conditions outlined below, please let me know as soon as possible so
we can come to an agreement about how to resolve the incomplete grade.

The PSU Bulletin States:

\begin{quote}
A student may be assigned an ``I'' grade by an instructor when all of
the following criteria apply:
\end{quote}

\begin{quote}
\begin{itemize}
\tightlist
\item
  Quality of work in the course up to that point must be C level or
  above.
\item
  Essential work remains to be done. ``Essential'' mens that a grade for
  the course could not be assigned without dorpping one or more grapde
  points below the level achievable upon completion of the work
\item
  Reasons for assigning an ``I'' must be acceptable to the instructor.
  The student does not have the right to demand an ``I''. The
  circumstances must be unforeseen or be beyond the control of the
  student. An instructor is entitled to insist on appropriate medical or
  other documentation. In no case is an ``Incomplete'' grade given to
  enable a student to do additional work to raise a deficient grade.
\end{itemize}
\end{quote}

\begin{quote}
A written agreement, signed by both the student and the instructor,
should include a statement of the remaining work to be done to remove
the ``I'' grade, and the date, not to exceed one year from the end of
the term of enrollment for the course, by which work must be completed
in order to earn credit toward the degree. The instructor must specify
the highest grade which may be awarded upon completion; the grade
awarded should not exceed the level of achievement attained during the
regular course period.
\end{quote}

\section{Plaigarism Policy}\label{plaigarism-policy}

Plagiarism will not be tolerated. It is a serious issue and is a
violation of the
\href{https://www.pdx.edu/dos/psu-student-code-conduct}{PSU Student
Conduct Code} For tips on how to recognize and avoid plagiarism, see the
resources available from the
\href{https://library.pdx.edu/diy/avoid-plagiarism}{PSU library}.

\section{Equitable and Safe Learning
Environment}\label{equitable-and-safe-learning-environment}

As an instructor, one of my responsibilities is to help create a safe
learning environment for my students and for the campus as a whole. We
expect a culture of professionalism and mutual respect in our department
and class. You may report any incident of discrimination or
discriminatory harassment, including sexual harassment, to either the
\href{https://www.pdx.edu/diversity/office-of-equity-compliance}{Office
of Equity and Compliance} or the \href{https://www.pdx.edu/dos/}{Office
of the Dean of Student Life}.

Please be aware that as a faculty member, I have the responsibility to
report any instances of sexual harassment, sexual violence and/or other
forms of prohibited discrimination. If you would rather share
information about sexual harassment or sexual violence to a confidential
employee who does not have this reporting responsibility, you can find a
list of those individuals. For more information about Title IX please
complete the required student module Creating a Safe Campus in your D2L.
\# Diversity and Inclusion

PSU values diversity and inclusion; we are committed to fostering mutual
respect and full participation for all students. My goal is to create a
learning environment that is equitable, useable, inclusive, and
welcoming. If any aspects of instruction or course design result in
barriers to your inclusion or learning, please notify me. The Disability
Resource Center (DRC) provides reasonable accommodations for students
who encounter barriers in the learning environment. If you have, or
think you may have, a disability that may affect your work in this class
and feel you need accommodations, contact the Disability Resource Center
to schedule an appointment and initiate a conversation about reasonable
accommodations. The DRC is located in 116 Smith Memorial Student Union,
503-725-4150, \url{drc@pdx.edu}, \url{https://www.pdx.edu/drc}.

\begin{itemize}
\item
  If you already have accommodations, please contact me to make sure
  that I have received a faculty notification letter and discuss your
  accommodations.
\item
  Students who need accommodations for tests and quizzes are expected to
  schedule their tests to overlap with the time the class is taking the
  test.
\item
  Please be aware that the accessible tables or chairs in the room
  should remain available for students who find that standard classroom
  seating is not useable.
\item
  For information about emergency preparedness, please go to the Fire
  and
  \href{https://www.pdx.edu/environmental-health-safety/fire-and-life-safety}{Life
  Safety webpage} for information
\end{itemize}




\end{document}

\makeatletter
\def\@maketitle{%
  \newpage
%  \null
%  \vskip 2em%
%  \begin{center}%
  \let \footnote \thanks
    {\fontsize{18}{20}\selectfont\raggedright  \setlength{\parindent}{0pt} \@title \par}%
}
%\fi
\makeatother